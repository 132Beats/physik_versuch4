\section{Theorie}
\subsection{Feder-Masse-Pendel}
Das Feder-Masse-Pendel besteht aus einer linear-elastischen Feder der Federhärte D, dessen Enden an einem festen Punkt und einem Massestück der Masse m befestigt ist.
Eine Auslenkung führt zu einer harmonischen Schwingung, da für die Feder das Hookesche Gesetz gilt:
\begin{equation}
F = -D_{0} \cdot \Delta l 
\end{equation}
\subsubsection{Schwingungsgleichung des Feder-Masse-Pendels}
Durch die Lösung der Differentialgleichung, die durch die Ersetzung der Kraft des Hookeschen Gesetzes durch den Newtonschen Ansatz, erhält man folgende Schwingungsgleichung:
\begin{equation}
        x(t) = sin(w_{0}\cdot t)
\end{equation}