\section{Theorie}
\subsection{Trägheitsmoment, statisch}
J =: Trägheitsmoment
\newline
m =: Masse
\newline
V =: Volumen
\newline
l =: Länge der Speichen
\newline
$R_{aussen}$ =: Außendurchmesser
\newline
$R_{innen}$ =: Innendurchmesser
\newline
Das Gesamtvolumen ist die Summe aus dem Rad(1 Hohlzylinder) und den 6 Speichen(3 Quader).
\begin{equation}
V_{ges} = V_{Hohlzylinder} + 3\cdot V_{Quader}
\end{equation}
Da von einer homogenen Dichteverteilung auszugehen ist, wird mit dem Volumenanteil und der die Masse der Einzelteile ausgerechnet.
\begin{equation}
M = \frac{M_{ges}\cdot V}{V_{ges}}
\end{equation}
Das gesamte Trägheitsmoment ist die Summe der einzelnen Trägheitsmomente.
\begin{equation}
J_{Hohlzylinder} = \frac{1}{2}M({R_{aussen}}^{2}+{R_{innen}}^{2})
\end{equation}
\begin{equation}
J_{Stange} = \frac{1}{3}M\cdot l^{2}
\end{equation}
\begin{equation}
J_{ges} = J_{Hohlzylinder}+3\cdot J_{Stange}
\end{equation}
\subsection{Trägheitsmoment, dynamisch}
$\omega$ =: Winkelgeschwindigkeit
\newline
$r_{s}$ =: Abstand Schwerpunkt zu Drehachse
\newline
m =: Masse des Fallrads
\newline
g =: Erdbeschleunigung
\newline
$J_{s}$ =: Trägheitsmoment durch die Schwerpunktsachse
\newline
T =: Periodendauer
\newline
\newline
Schwingungsgleichung für das Physikalische Pendel:
\begin{equation}
\omega = \sqrt{\frac{r_{s}\cdot m\cdot g}{J_{s}+m\cdot r_{s}^{2}}}
\end{equation}
Nach der Benutzung der Formel
\begin{equation}
\omega = \frac{2\pi}{T}
\end{equation}
und der Umstellung zum Trägheitsmoment (durch die Schwerpunktsachse) ergibt sich:
\begin{equation}
J_{s} = \frac{g\cdot m\cdot r_{s} \cdot T^{2}}{4\pi^{2}}
\end{equation}