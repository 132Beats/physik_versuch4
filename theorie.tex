\section{Theorie}
\subsection{Feder-Masse-Pendel}
Das Feder-Masse-Pendel besteht aus einer linear-elastischen Feder der Federhärte D, dessen Enden an einem festen Punkt und einem Massestück der Masse m befestigt ist.
Eine Auslenkung führt zu einer harmonischen Schwingung, da für die Feder das Hookesche Gesetz gilt:
\begin{equation}
F = -D_{0} \cdot \Delta l 
\end{equation}
\subsubsection{Schwingungsgleichung des Feder-Masse-Pendels}
Durch die Lösung der Differentialgleichung, die durch die Ersetzung der Kraft des Hookeschen Gesetzes durch den Newtonschen Ansatz, erhält man folgende Schwingungsgleichung:
\begin{equation}
        x(t) = sin(w_{0}\cdot t)
\end{equation}
Daraus ergibt sich die Periodendauer:

\begin{equation}
    T_{0} = \frac{2\pi}{w_{0}} = 2\pi \sqrt{\frac{m}{D_{0}}}
\end{equation}
\subsection{Mathematisches Pendel}
Für den Fall geringer Auslenkungs-Winkel gilt die Bewegungsgleichung:
\begin{equation}
	\frac{\partial ^2 \phi}{\partial t^2} = -\frac{g\cdot \phi}{l}
\end{equation}
\subsubsection{Schwingungsgleichung des Mathematischen Pendels}
Nach Auflösung dieser Differentialgleichung ergibt sich die Periodendauer:
\begin{equation}
T_{0} = 2\pi \sqrt{\frac{l}{g}}
\end{equation}
\iffalse
Diese Binominalverteilung lässt sich übertragen auf das Galton-Brett, wobei $x$ die Verteilung der Ereignisse auf die jede Kollision der Kugel mit einem Stahlstift im Brett darstellt. Hier wird $n$ als die aktuelle Reihe angesehen und $k$ ist die Stelle des vorhergehenden Stiftes in der horizontalen. Übertragen auf den Binominalkoeffizienten lässt sich folgende Formel herleiten:
\begin{equation}
    x=\binom{n-1}{k}=\frac{(n-1)!}{k!(n-1-k)!}
\end{equation}
Würde man $x$ für jede Kollision berechnen, ergibt sich ein Pascalsches Dreieck:

HIER KÖNNTE IHRE WERBUNG STEHEN!!! (oder auch nur ein Bild)

Die Wahrscheinlichkeit für jede Kollision in der Reihe $n$ lässt sich berechnen, indem $x_{Position}$ durch die Summe aller $x$ in Reihe $n$ geteilt wird:

\begin{equation}
    p_{Treffer}=\frac{x_{Position}}{\sum_{i=0}^n x_i}
\end{equation}

\subsubsection{Erwartung an die Durchführung}
Die Erwartung an die Versuchsdurchführung mit dem Galton-Brett ist, dass sich die Kugeln in der letzten Reihe (Auffangbehälter) binominal verteilen. Das Galton-Brett im Versuch hat 14 Reihen (=14 Auffangbehälter), wodurch sich folgende, optimale Verteilung ergeben sollte:

\begin{table}[thb]
    \centering
    \csvloop{
      file=tables/optimale_binomvert.csv,
      no head,
      column count=15,
      before reading=\begin{tabular}{l|l|l|l|l|l|l|l|l|l|l|l|l|l|l},
      /csv/separator = semicolon,
      command = \csvlinetotablerow,
      late after line=\\\hline,
      late after first line=\\\hline,
      late after last line=\\,
      respect percent = true,
      after reading=\end{tabular}
    }
    \caption{Erwartungswert am Galton-Brett}
  \end{table}

\subsection{Freier Fall}
Beim Versuch des Freien Falles wird die Zeit gestoppt, die ein Massekörper benötgt, eine Distanz von 86,2cm im freien Fall zu überwinden. Hierbei wird die Zeit gemessen, die der Massekörper benötgt, diese Distanz zu überwinden.
Ziel des Versuches ist es, Messungenauigkeiten herauszufinden und die entsprechenden statistischen Fehler zu berechnen.

\subsubsection{Physikalischer Hintergrund}
Beim freien Fall wird eine gleichförmige Beschleunigung von $g=9,81\frac{m}{s^{2}}$ angenommen. Durch das Weg-Zeit-Gesetz bildet sich folgende Formel, durch welche sich die Strecke berechnen lässt:
\begin{equation}
    s(t)=\frac{1}{2}g*t^{2}
\end{equation}
Umgeformt nach $t$:
\begin{equation}
    t_{Fall}=\sqrt{\frac{2h}{g}}
\end{equation}
$h$ repräsentiert die Fallhöhe in $m$ (Meter).

\subsubsection{Berechnung der idealen Fallzeit}
Im Versuch beträgt die Fallhöhe 856mm (+- 1mm), was 0,856m entspricht. Hieraus ergibt sich in die obige Formel eingesetzt:
\begin{equation}
    t_{Fall}=\sqrt{\frac{2*0,856m}{9,81\frac{m}{s^{2}}}}=0,418s
\end{equation}
Der Erwartungswert der Messung liegt im Idealfall somit bei 0,418s, was 418ms entspricht.
\fi
