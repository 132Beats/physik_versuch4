\section{Durchführung der Messung}

\subsection{Längenmessungen am Fallrad}
Während des Versuchs wurden die verschiedener Längen, Breiten und Tiefen des Fallrads notiert und 5 Mal wiederholt.
Die Masse wurde auf 517(+-1) Gramm gewogen.
\newline
\newline
l(i) =: Länge in der i-ten Messung
\newline
m =: Mittelwert der Messungen
\newline
s =: Standardabweichung der Messungen
\newline
s(m) =: Standardabweichung des Mittelwerts
\begin{table}[thb]
  \centering
  \hline
  \csvloop{
    file=tables/maxmess.csv,
    no head,
    column count=9,
    before reading=\begin{tabular}{|l|l|l|l|l|l|l|l|l|l|l|l|l|l|l|l|l|},
    /csv/separator = semicolon,
    command = \csvlinetotablerow,
    late after line=\\,
    late after first line=\\\hline,
    late after last line=\\\hline,
    respect percent = true,
    after reading=\end{tabular}
  }
  %\caption{Alle Zeitmessungen in Sekunden}
\end{table}

\subsection{Zeitmessungen des schwingenden Rades}
In diesem Experiment wurden 5 Mal 20 Schwingungen des Rades, während es auf einer Schneide hing, gemessen.
\newline
\newline
n =: Anzahl der Perioden
\newline
t(i) =: Zeit von n Perioden im i-ten Versuch
\newline
m =: Mittelwert der Zeitmessungen
\newline
s =: Standardabweichung der Zeitmessungen
\newline
s(m) =: Standardabweichung des Mittelwerts
\begin{table}[thb]
	\centering
	\hline
	\csvloop{
		file=tables/maxtime.csv,
		no head,
		column count=9,
		before reading=\begin{tabular}{|l|l|l|l|l|l|l|l|l|l|l|l|l|l|l|l|l|},
			/csv/separator = semicolon,
			command = \csvlinetotablerow,
			late after line=\\,
			late after first line=\\\hline,
			late after last line=\\\hline,
			respect percent = true,
			after reading=\end{tabular}
	}
%\caption{Alle Zeitmessungen in Sekunden}
\end{table}

%\csvautolongtable[
%  /csv/separator = semicolon,
%]{tables/messwerte_freier_fall.csv}