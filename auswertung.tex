\newpage
\section{ Auswertung der Messergebnisse und Fehlerrechnung}

\subsection{Statische Errechnung des Trägheitmoments}
Gesucht ist das Trägheitsmoment.
\newline
J =: Trägheitsmoment
\newline
m =: Masse
\newline
V =: Volumen
\newline
$a_{innen}$ =: Innendurchmesser
\newline
$a_{aussen}$ =: Außendurchmesser
\newline
$a_{speiche}$ =: Breite der Speichen
\newline
$c_{speiche}$ =: Tiefe der Speichen
\newline
$c_{rad}$ =: Tiefe der Rades
\newline
$a_{rad}$ =: Breite der Rades
\newline
Das Gesamtvolumen ist die Summe aus dem Rad(1 Hohlzylinder) und den 6 Speichen(3 Quader). Bei Summen wird der absolute Fehler der Summanden und bei Produkten der relative Fehler der Faktoren addiert.
\begin{equation}
V = (\pi \cdot (\frac{a_{aussen}}{2})^{2}\cdot c_{rad}-\pi \cdot (\frac{a_{innen}}{2})^{2}\cdot c_{rad})+3\cdot (a_{aussen}\cdot a_{speiche}\cdot c_{speiche})
\end{equation}
Eingesetzt ergibt sich ein Volumen von 188600(+-1300) $mm^{3}$. Das Rad hat ein Volumen von 171100(+-1100) $mm^{3}$ und die 3 Quader jeweils 5842(+-64) $mm^{3}$. Da von einer homogenen Dichteverteilung auszugehen ist, kann mit dem Volumenanteil und der Gesamtmasse (0,5170(+-0,0001))kg die Masse der Einzelteile ausgerechnet werden.
\begin{equation}
M = \frac{M_{ges}\cdot V}{V_{ges}}
\end{equation}
Das Rad hat eine Masse von 0,518(+-0,007) kg und 3 Quader jeweils eine Masse von 0,0259(+-0,0005) kg.
\newline
Nun können die Trägheitsmomente der Einzelteile bestimmt werden.
\begin{equation}
J_{Hohlzylinder} = \frac{1}{2}M((\frac{a_{aussen}}{2})^{2}+(\frac{a_{innen}}{2})^{2})
\end{equation}
Das Trägheitsmoment des Hohlzylinders beträgt 1756(+-25) $kg\cdot mm^{2}$
\begin{equation}
J_{Stange} = \frac{1}{3}M\cdot a_{innen}^{2}
\end{equation}
Das Trägheitsmoment einer homogenen Stange (zwei Speichen) beträgt 86,1(+-1,8) $kg\cdot mm^{2}$
\newline
Das gesamte Trägheitsmoment ist die Summe der einzelnen Trägheitsmomente.
\begin{equation}
J_{ges} = J_{Hohlzylinder}+3\cdot J_{Stange}
\end{equation}
Das gesamte Trägheitsmoment beträgt  2014(+-30) $kg\cdot mm^{2}$.
%\begin{table}[thb]
%	\centering
%	\hline
%	\csvloop{
%		file=tables/D1.csv,
%		no head,
%		column count=3,
%		before reading=\begin{tabular}{|l|l|l|l|l|l|l|l|l|l|l|l|l|l|l|l|l|},
%			/csv/separator = semicolon,
% 			command = \csvlinetotablerow,
%			late after line=\\,
%			late after first line=\\\hline,
%			respect percent = true,
%			after reading=\end{tabular}
%	}
%\caption{Alle Zeitmessungen in Sekunden}
%\end{table}



%\subsection{Errechnung des Abrollradius}




\subsection{Dynamische Errechnung des Trägheitmoments}

$\omega$ =: Winkelgeschwindigkeit
\newline
$r_{s}$ =: Abstand Schwerpunkt zu Drehachse
\newline
m =: Masse des Fallrads
\newline
g =: Erdbeschleunigung
\newline
$J_{s}$ =: Trägheitsmoment durch die Schwerpunktsachse
\newline
T =: Periodendauer
\newline
\newline
Schwingungsgleichung für das Physikalische Pendel:
\begin{equation}
\omega = \sqrt{\frac{r_{s}\cdot m\cdot g}{J_{s}+m\cdot r_{s}^{2}}}
\end{equation}
Nach der Benutzung der Formel
\begin{equation}
\omega = \frac{2\pi}{T}
\end{equation}
und der Umstellung zum Trägheitsmoment (durch die Schwerpunktsachse) ergibt sich:
\begin{equation}
J_{s} = \frac{g\cdot m\cdot r_{s} \cdot T^{2}}{4\pi^{2}}
\end{equation}
Das Trägheitsmoment beträgt 2806(+-25) $kg\cdot mm^{2}$.
