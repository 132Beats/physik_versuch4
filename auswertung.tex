\newpage
\section{ Auswertung der Messergebnisse und Fehlerrechnung}

\subsection{Statische Errechnung des Trägheitmoments}
Gesucht ist das Trägheitsmoment.
\newline
J =: Trägheitsmoment
\newline
m =: Masse
\newline
g =: Erdbeschleunigung
\newline
l =: Auslenkung
\begin{equation}
D = \frac{m\cdot g}{l}
\end{equation}
Da die Formel nur eine mit Fehlern behaftete Größe beinhaltet (die Auslenkung l), entspricht der relative Fehler der Federhärte dem relativen Fehler der Auslenkung.


\begin{table}[thb]
	\centering
	\hline
	\csvloop{
		file=tables/D1.csv,
		no head,
		column count=3,
		before reading=\begin{tabular}{|l|l|l|l|l|l|l|l|l|l|l|l|l|l|l|l|l|},
			/csv/separator = semicolon,
			command = \csvlinetotablerow,
			late after line=\\,
			late after first line=\\\hline,
			late after last line=\\\hline,
			respect percent = true,
			after reading=\end{tabular}
	}
	%\caption{Alle Zeitmessungen in Sekunden}
\end{table}
\subsection{Errechnung des Abrollradius}

\subsection{Dynamische Errechnung des Trägheitmoments}
Schwingungsgleichung für das Physikalische Pendel:
\begin{equation}
\omega = \sqrt{\frac{r_{s}\cdot m\cdot g}{J_{s}+m\cdot r_{s}^{2}}}
\end{equation}
Nach der Benutzung der Formel
\begin{equation}
\omega = \frac{2\pi}{T}
\end{equation}
und der Umstellung zum Trägheitsmoment (durch die Schwerpunktsachse) ergibt sich:
\begin{equation}
J_{s} = \frac{g\cdot m\cdot r_{s} \cdot T^{2}}{4\pi^{2}}
\end{equation}
